\documentclass[10pt,twocolumn,letterpaper]{article}

\usepackage{cvpr}
\usepackage{times}
\usepackage{epsfig}
\usepackage{graphicx}
\usepackage{amsmath}
\usepackage{amssymb}

% Include other packages here, before hyperref.

% If you comment hyperref and then uncomment it, you should delete
% egpaper.aux before re-running latex.  (Or just hit 'q' on the first latex
% run, let it finish, and you should be clear).
\usepackage[breaklinks=true,bookmarks=false]{hyperref}

\cvprfinalcopy % *** Uncomment this line for the final submission

\def\cvprPaperID{****} % *** Enter the CVPR Paper ID here
\def\httilde{\mbox{\tt\raisebox{-.5ex}{\symbol{126}}}}

% Pages are numbered in submission mode, and unnumbered in camera-ready
%\ifcvprfinal\pagestyle{empty}\fi
%\setcounter{page}{4321}
\begin{document}

%%%%%%%%% TITLE
\title{Silhouette-based live 3D reconstruction}

\author{Stefan Goetschi\\
ETH Zurich\\
{\tt\small gostefan@ethz.ch}
% For a paper whose authors are all at the same institution,
% omit the following lines up until the closing ``}''.
% Additional authors and addresses can be added with ``\and'',
% just like the second author.
\and
Enes Poyraz\\
ETH Zurich\\
{\tt\small poyraze@ethz.ch}
}

\maketitle
%\thispagestyle{empty}

%%%%%%%%% ABSTRACT
\begin{abstract}
   The ABSTRACT is to be in fully-justified italicized text, at the top
   of the left-hand column, below the author and affiliation
   information. Use the word ``Abstract'' as the title, in 12-point
   Times, boldface type, centered relative to the column, initially
   capitalized. The abstract is to be in 10-point, single-spaced type.
   Leave two blank lines after the Abstract, then begin the main text.
   Look at previous CVPR abstracts to get a feel for style and length.
\end{abstract}

%%%%%%%%% BODY TEXT
\section{Introduction}

In the following we will introduce in all the concepts we used in our project. This ranges from Grab Cut over 3D reconstruction on to volume rendering.

Further we would like to shortly introduce all libraries we used for our implementation.

%-------------------------------------------------------------------------
\subsection{Libraries}

Describe the libraries.

%-------------------------------------------------------------------------
\subsection{Grab Cut}

Describe Grab Cut.

\subsection{3D Reconstruction}

Describe 3D reconstruction.

%-------------------------------------------------------------------------
\subsection{Volume Rendering}

Describe volume rendering.

\section{The Application}

Now that all the necessairy knowledge is introduced we would like to go into some parts of our project in depth.

%-------------------------------------------------------------------------
\subsection{Class diagram}

Here we state which class is used in which state and how they communicate with each other

%-------------------------------------------------------------------------
\subsection{Grab Cut selection}

This explains how the strokes are captured and relayied to the native code

%-------------------------------------------------------------------------
\subsection{Matrix handling}

Here we explain how the matrix contents are gathered and used

%-------------------------------------------------------------------------
\subsection{Texture usage}

Here we explain the concept of using a 2D texture to simulate a 3D texture

%------------------------------------------------------------------------
\section{Conclusions}

%-------------------------------------------------------------------------
\subsection{Advantages of a mobile solution}

We blable about how nice it is that it is all on the phone now... yay!

%-------------------------------------------------------------------------
\subsection{Possible Improvements}

We tell them what part we screwed up (Multithreading, etc.) As in the presentation.

%-------------------------------------------------------------------------
\subsection{Limitations}

We talk about the problem that only the visual hull can be reconstructed using this technique and that voxel carving would be so much nicer but probably not feasible.


{\small
\bibliographystyle{ieee}
\bibliography{egbib}
}

\end{document}
